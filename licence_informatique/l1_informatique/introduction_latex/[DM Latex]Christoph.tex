\documentclass[12pt,a4paper]{article}
\usepackage[utf8]{inputenc}
\usepackage[T1]{fontenc}
\usepackage[french]{babel}
\usepackage{textcomp}
\usepackage{array,multirow,makecell}
\usepackage{lmodern}
\usepackage{geometry}
\usepackage{graphicx}
\usepackage{xcolor}
\usepackage{titlesec}
\usepackage{titletoc}
\usepackage{fancyhdr}
\usepackage{enumitem}
\usepackage{hyperref}
\hypersetup{pdfstartview=XYZ}
\frenchbsetup{StandardLists=true}
\geometry{hmargin=2.5cm,vmargin=2cm}
\pagestyle{empty}
\addto\captionsfrench{\renewcommand{\contentsname}{Sommaire}}
\begin{document}
\begin{titlepage}
 \fontfamily{phv}\selectfont
 \vspace*{\stretch{1}}
 \begin{flushright}\LARGE
   Christoph Samuel\\Coig Maxime
 \end{flushright}
 \hrule
 \begin{flushleft}\huge\bfseries
   La programmation informatique
 \end{flushleft}
 \vspace*{\stretch{2}}
 \begin{center}
    Avril 2021
 \end{center}
\end{titlepage}
\addto\captionsfrench{\renewcommand{\abstractname}{La programmation 
informatique}}
\setcounter{tocdepth}{3}
\clearpage
 \tableofcontents
 \listoffigures
 \listoftables
\vspace{2,5cm}
\begin{abstract}
 Le document porte sur la programmation informatique, dans un premier temps nous
 verrons une bréve histoire sur la programmation\ref{histoire} ensuite nous 
 verrons ses différentes pratiques\ref{pr}, pour enfin finir sur la création un 
 programme informatique et les différents types de programmation\ref{3} dans 
 l'informatique.
\end{abstract}
\newpage
\section{Introduction}
La programmation dans le domaine informatique est l'ensemble des 
activités qui permettent l'écriture des programmes informatiques. 
C'est une étape importante de la conception de logiciel (voire de 
matériel).Pour écrire le résultat de cette activité, on 
utilise un \textit{langage de programmation\footnote{
notation conventionnelle destinée à formuler des algorithmes et 
produire des programmes informatiques qui les appliquent}.}\\
\begin{figure}[!h]
 \centering
  \includegraphics[height=100px, width=160px]{intro}
  \caption{Programmation informatique}
  \label{fig:introduction}
\end{figure}

La programmation représente le plus souvent le codage, c’est-à-dire 
la rédaction du code source d'un logiciel. On utilise plutôt le 
terme développement pour lister l'ensemble des activités liées 
à la création d'un logiciel.
\section{\label{histoire}Une brève histoire de la programmation}
La première machine programmable (c’est-à-dire machine dont les
possibilités changent quand on modifie son "programme") est
probablement le métier à tisser de Jacquard, qui a été réalisé en
1801. La machine utilisait une suite de cartons perforés. Les trous
indiquaient le motif que le métier suivait pour réaliser un
tissage.\\
\begin{figure}[!h]
 \centering
  \includegraphics[height=215px, width=140px]{machine.jpg}
  \caption{Machine métier a tisser}
  \label{fig:Métier a tisser}
\end{figure}
\newpage
Cette innovation a été ensuite améliorée par Herman Hollerith d'IBM
pour le développement de la fameuse carte perforée d'IBM.\\
\begin{figure}[ht]
 \centering
  \includegraphics[height=160px, width=120px]{dessin.jpg}
  \caption{Métier a tisser version dessin}
  \label{fig:Métier a tisser version dessin}
\end{figure}

En 1936, la publication de On Computable Numbers with an Application
to the Entscheidungsproblem\cite{livre2}\footnote{article fondateur de la 
science informatique} par Alan Mathison Turing donne le coup  d'envoi à la 
création de l'ordinateur programmable. Il y présente sa machine de 
Turing, le premier calculateur universel programmable, et invente les
concepts et les termes de programmation et de programme.
Les programmes devenant plus complexes, cela est devenu presque 
impossible, parce qu'une seule erreur rendait le programme entier 
inutilisable. Avec les progrès des supports de données, il devient 
possible de charger le programme à partir de cartes perforées, 
contenant la liste des instructions en code binaire spécifique à un 
type d'ordinateur particulier. La puissance des ordinateurs
augmentant, on les utilisa pour faire les programmes, les programmeurs préférant
naturellement rédiger du texte plutôt que des suites de 0 et de 1, à 
charge pour l'ordinateur d'en faire la traduction lui-même. Avec le 
temps, de nouveaux langages de programmation sont apparus, faisant de
plus en plus abstraction du matériel sur lequel devaient tourner les 
programmes. Ceci plusieurs facteurs de gains : ces langages sont plus
faciles à apprendre, un programmeur peut produire du code plus 
rapidement, et les programmesproduits peuvent tourner sur différents 
types de machines.
\subsection{La fin des programmeurs ?}
De tous temps, on a prédit « la fin des programmeurs ». Dans les 
années 60, les langages symboliques comme AUTO-CODE, Cobol et Fortran
ont  en effet mis fin "en grande partie" à la programmation de bas 
niveau tel que l'assembleur\footnote{langage de plus bas niveau qui 
représente le langage machine sous une forme lisible par un humain.}.
Il semblait alors clair que n'importe qui était capable d'écrire du 
code du type:
\begin{center}
\begin{figure}[ht]
 \centering
  \includegraphics[height=40px, width=450px]{exemple1}
\end{figure}
  ou
\newpage
\begin{figure}[ht]
  \centering
  \includegraphics[height=25px, width=400px]{exemple2}
\end{figure}  
  plutôt que des dizaines de lignes comme\\
\begin{figure}[ht]
\centering
  \includegraphics[height=60px, width=450px]{exemple3}
\end{figure}
\end{center}         
Pourtant il a vite fallu se rendre compte que la programmation ne se
limitait pas au codage, et que la conception d'applications était un
vrai métier qui ne s'improvise pas.\\
\section{\label{pr}Pratique}
En programmation informatique la pratique est établie sos plusieurs 
points parmis lesquel on retrouve l'algorithmique, la gestion de 
versions, l'optimisation du code, la programmation système, le 
refactoring et des tests (intégration et unitaire).
 \begin{enumerate}[leftmargin=30px]
    \item L'algorithmique est l'étude et la production de règles et 
    techniques impliquées dans la définition et la conception 
    d'algorithmes.
    \item La gestion de versions consiste à gérer l'ensemble des 
    versions d'un ou plusieurs fichiers, elle concerne surtout la 
    gestion des codes source. 
    \item  L'optimisation de code consiste à améliorer l'efficacité 
    du code informatique d'un programme (plus rapide, moins de place 
    mémoire...)
    \item  La programmation système est un type de programmation qui 
    vise au développement de programmes qui font partie du système 
    d’exploitation d’un ordinateur ou qui en réalisent les fonctions.
    \item  Le refactoring \footnote{réusinage de code} est 
    l'opération consistant à retravailler le code source d'un 
    programme informatique sans y ajouter des fonctionnalités de 
    façon à en améliorer la lisibilité.
    \item  Le test d'intégration est une phase de tests,vérifiant le 
    bon fonctionnement d'une partie précise d'un logiciel.
 \end{enumerate}
\section{Phases de création d'un programme}
On retrouve dans la programmation informatique une étape crutiale qui
est la création d'un programme. Cette création s'effectue en 4 
phases, une phase de conception suivie d'une phase de codage puis une
phase de transforation du code source et pour finir le test du 
programme final.
\newpage
\subsection{Conception}
La phase de conception définit le but du programme. Si on fait une 
rapide analyse fonctionnelle d'un programme, on détermine 
essentiellement les données qu'il va traiter (données d'entrée), la 
méthode employée (l'algorithme), et le résultat (données de 
sortie). Les données d'entrée et de sortie peuvent être de 
nature très diverses. On retrouve 
en général les mêmes fonctionnalités de base :\\

\textbf{\underline{Pour la programmation impérative}}\\
\begin{table}[!h]
\centering
 \begin{tabular}{|c|c|c|} \\ \hline
  Si alors & Tant que & Pour \\ \hline
  SI prédicat         & TANT QUE prédicat   & Pour variable allant de  \\ 
      &               & borne inférieur à borne supérieur \\ 
    ALORS faire ceci  &          faire...   &                           \\ 
    SINON faire cela  &                     &                      faire ...\\ 
    \hline
 \end{tabular}
 \caption{Les éléments les plus important dans la programmation informatique}
 \label{1}
\end{table}
\subsection{Codage}
Une fois l'algorithme défini, l'étape suivante est de coder le 
programme. Le codage dépend de l'architecture sur laquelle va 
s'exécuter le programme, de compromis temps-mémoire, et d'autres 
contraintes. Ces contraintes vont déterminer quel langage de 
programmation utiliser pour "convertir" l'algorithme en code source.\\
\begin{figure}[ht]
 \centering
  \includegraphics[height=60px, width=130px]{codage.jpg}
  \caption{Le codage}
  \label{fig:Le codage}
\end{figure}
\subsection{Transformation du code source}
Le code source\footnote{texte qui présente les 
instructions composant un programme sous une forme lisible} n'est (presque) 
jamais utilisable tel quel. Il est 
généralement écrit dans un langage "de haut niveau", compréhensible 
pour l'homme, mais pas pour la machine.\cite{article1}
\subsubsection{Compilation}
Certains langages sont ce qu'on appelle des langages compilés\cite{article2}. En
toute généralité, la compilation est l'opération qui consiste à transformer 
un langage source en un langage cible. Dans le cas d'un programme, le 
compilateur va transformer tout le texte représentant le code source du
programme, en code compréhensible pour la machine, appelé code machine.Dans 
le cas de langages dits compilés, ce qui est exécuté est le résultat de la 
compilation. Une fois effectuée, l'exécutable obtenu peut être utilisé sans 
le code source.\\
\newpage
\begin{figure}[ht]
 \centering
  \includegraphics[height=180px, width=240px]{compilation.jpg}
  \caption{Exemple de compilation}
  \label{fig:Exemple de compilation}
\end{figure}
Il faut également noter que le résultat de la compilation n'est pas 
forcément du code machine correspondant à la machine réelle, mais peut 
être du code compris par une machine virtuelle (c'est-à-dire un 
programme simulant une machine), auquel cas on parlera de bytecode\footnote{code
intermédiaire entre les instructions machines et le code source, qui n'est pas 
directement exécutable}. 
C'est par exemple le cas en Java\footnote{Java est un langage de programmation 
orienté objet}. L'avantage est que, de cette façon, 
un programme peut fonctionner sur n'importe quelle machine réelle, du 
moment que la machine virtuelle existe pour celle-ci.\\
\subsubsection{Interprétation}
D'autres langages ne nécessitent pas de phase spéciale de compilation. 
La méthode employée pour exécuter le programme est alors différente. Le
programme entier n'est jamais compilé. Chaque ligne de code est 
compilée « en temps réel » par un programme. On dit de ce programme 
qu'il interprète le code source. Par exemple, python ou perl sont des 
langages interprétés.\\
\begin{figure}[ht]
 \centering
  \includegraphics[height=150px, width=280px]{code_python.png}
  \caption{Un exemple de programmation en Python}
  \label{fig:Un exemple de programmation en Python}
\end{figure}
\newpage
Cependant, ce serait faux de dire que la compilation n'intervient pas. 
L'interprète produit le code machine, au fur et à mesure de l'exécution
du programme, en compilant chaque ligne du code source.\\
\begin{table}
\centering
 \begin{tabular}{|c|c|c|c|c|} \\ \hline
  langage de programmation & rang 2020 & rang 2015 & rang 2010 & rang 2005 \\
  \hline
  Java    & 1 & 2 & 1 & 2 \\ \hline
  C       & 1 & 1 & 2 & 1 \\ \hline
  Python  & 3 & 7 & 6 & 6 \\ \hline
  C++     & 4 & 4 & 4 & 3 \\ \hline
 \end{tabular}
  \caption{Classement des meilleurs languages de programmation depuis 2005}
  \label{2}
  \end{table}
\subsubsection{Avantages, inconvénients}
Ces langages comportent tous des avantages, des inconvénients et leur domaines 
de pratique on retrouve donc des études et ouvrages pour chaqu'un d'eux par 
exemple avec The C programming language\cite{livre1} pour le C.
Les avantages généralement retenus pour l'utilisation de langages 
"compilés", est qu'ils sont plus rapides à l'exécution que des langages
interprétés, car l'interprète doit être lancé à chaque exécution du
programme, ce qui mobilise systématiquement les ressources.\\
Traditionnellement, les langages interprétés offrent en revanche une 
certaine portabilité (la capacité à utiliser le code source sur 
différentes plates-formes), ainsi qu'une facilité pour l'écriture du 
code. En effet, il n'est pas nécessaire de passer par la phase de 
compilation pour tester le code source. Les quatres languages\ref{2} vu 
precédements sont les plus utilisés.
\subsection{Test du programme}
C'est l'une des étapes les plus importantes de la création d'un 
programme. En principe, tout programmeur se doit de vérifier chaque 
partie d'un programme, de le tester. Il existe différents types de 
test. \\
\textbf{On peut citer en particulier :}
\begin{enumerate}
 \item Test unitaire
 \item Test d'intégration
 \item Test de performance
\end{enumerate}
\begin{figure}[ht]
 \centering
  \includegraphics[height=100px, width=400px]{schema.jpg}
  \caption{Schéma bilan sur la transformation du code source}
  \label{Schéma bilan sur la transformation du code source}
\end{figure}
\newpage
Il convient de noter qu'il est parfois possible de vérifier un programme 
informatique, c'est-à-dire prouver, de manière plus ou moins automatique, qu'il 
assure certaines propriétés.\\
\section{Techniques de programmation}
On retrouve de nombreuses techniques de programmation par exemple la 
programmation par objet qui consiste à utiliser des techniques de programmation 
pour mettre en œuvre une conception basée sur les objets. Celle-ci peut être 
élaborée en utilisant des méthodologies de développement logiciel objet, la 
programmation  concurrente qui est un paradigme de programmation tenant compte, 
dans un  programme, de l'existence de plusieurs piles sémantiques qui peuvent 
être appelées threads, processus ou tâches. Elles sont matérialisées en machine 
par une pile d'exécution et un ensemble de données privées, ou encore comme la 
programmation impérative\ref{1} vu précédement.\\
\textbf{Voici une liste de quelques autres techniques de programmation que l'on 
retrouve dans la programmation informatique}
\begin{itemize}[leftmargin=50px]
  \item Programmation fonctionnelle
  \item Programmation logique
  \item Programmation orientée prototype
  \item Programmation par contraintes \cite{the}
  \item Programmation structurée
  \label{3}
\end{itemize}
\newpage  
\bibliographystyle{plain}
\bibliography{ref.bib}
\end{document}